\setlist{leftmargin=*,itemsep=2pt,parsep=0pt}
\renewcommand{\labelitemi}{--}
\setlength{\tabcolsep}{.75em}
%
\begin{uvp}      
       	\Einstieg{08:045 -- 08:50}{5 Min.}{%
			%Begrüßung/Vorstellung der der Gäste\newline
			L. knüpft an die letzte Stunde (Caesar) an und nennt die Fragestellung für die heutige Stunde: 
			\textit{Wie kann das Problem der Schlüsselübermittlung gelöst werden? }
			}%
       		{
       		Die SuS rufen sich die Problematik der letzten Stunde in Gedächtnis	
       		}
       		{
       		Plenum/Beamer, Lehrervortrag, LS-Gespräch
       		}%
		%               
		\Gelenk{\small \textit{Um...}}   
        \Erarbeitung{08:50 -- 09:05}{15 Min.}{% 
			\textbf{Entwickeln und Testen des Verfahrens:} \newline
			L. teilt Material aus und teilt die SuS in Gruppen auf. L. erläutert die
			Aufgabenstellung des Rollenspiels. Während des Spiels versucht L, die übermittelte 
			Nachricht zu lesen. Dies kann durch die SuS nur verhindert werden, wenn sie ein 
			geschicktes Schlüsselaustauschverfahren nutzen.
       		}%
       		{Die in Gruppen aufgeteilten SuS entwickeln ein Verfahren, bei dem sie mittels
			ihrer Schlösser und einem  Briefumschlag eine Nachricht sicher übermitteln können. Sie
			führen das Verfahren anschließend durch.}%
       		{2 Vorhängeschlösser u. 1 Briefumschlag (jeweils), Gruppenarbeit}
       	 \Sicherung{09:05 -- 09:10}{5 Min.}{% 
			\textbf{Demonstration eines man-in-themiddle-Angriffs:}
        		L. fordert die SuS auf, das Szenario noch einmal durchzuspielen, greift die Nachrichtenübermittlung
        		an, indem er/sie 
        		\begin{enumerate}
        		 	\item  die Nachricht abfängt, mit einem eigenen Schloss versieht und wieder zurückschickt und
				\item  den Umschlag austauscht und weiter schickt
			\end{enumerate}      
       		}%
       		{Entscheiden sich die SuS für das DiffieHellman-Verfahren, so führen Sie das Verfahren
        		erneut durch und beobachten dabei, wie die Übermittlung der Nachricht durch L. angegriffen wird.}%
       		{Rollenspiel, \newline LS-Gespräch}
        \Gelenk{\small \textit{Beendet...}} 
        \Erarbeitung{09:10 -- 09:20}{10 Min.}{%
			L. erweitert die Situationsbeschreibung um die Möglichkeit, dass die Gruppen die Kommunikation
			in einem einmaligen Treffen vorbereiten. 
       		}%
       		{ Die SuS erarbeiten, ggf. durch geeignete Einhilfen von L, ein Verfahren mit vorangehendem Austausch der geöffneten Schlösser.}%
       		{Gruppenarbeit, anschließend Plenum}%     
       	\Sicherung{09:20 -- 09:10}{10 Min.}{%
			L bittet eine(e) Schüler(in) die Schrittfolge des Verfahrens an der Tafel festzuhalten. 
			L fragt nach der zentralen Idee des gewählten Lösungsansatzes.
       		}%
       		{Ein(e) Schüler(in) hält die Schrittfolge des Verfahrens an der Tafel fest, die anderen SuS
    		unterstützen ihn / sie dabei. Die SuS halten die Schrittfolge auf dem AB fest. Die SuS benennen
    		die Trennung von Schließ- und Öffnungsfunktion als Lösung.}%
       		{Schülervortrag / Tafel}%     
\end{uvp}
