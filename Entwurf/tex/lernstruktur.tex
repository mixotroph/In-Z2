\newcolumntype{Y}{>{\small\arraybackslash}X}
\begin{longtable}{>{\small\raggedright}lYYYYY}
\toprule
% ################################################################################################################################################
%\textbf{Pha\-se} &
\rotatebox[origin=rB]{90}{\textbf{Phase}} & 
\raggedright\textbf{Didaktische oder methodische Entscheidungen} & \raggedright\textbf{Begründung bezogen auf den fachl.-inhalt. Schwerpunkt} &
	\raggedright\textbf{Begründung bezogen auf die SuS (inkl. Lernpsychologie)} & \raggedright\textbf{Mögliche Schwierigkeiten} 
	& {\raggedright\textbf{Vorbeugende Maß\-nah\-men/mögliche Alternativen}}\\
\midrule
% ################################################################################################################################################
%\textbf{Einstieg} 
\rotatebox[origin=rB]{90}{\textbf{Einstieg}} & 
%-------------------------------------------------------------------------------------------------------------------------------------------------
Informierender Einstieg: Transparenz durch Vorstellung des Stundenthemas. Festlegung von Fragestellung im Plenum, um einheitliche
Ausgangslage zu schaffen. &
%-------------------------------------------------------------------------------------------------------------------------------------------------
Die Einstiegsphase wird zu Gunsten der Erarbeitungsphase bewusst kurz gehalten. &
%-------------------------------------------------------------------------------------------------------------------------------------------------
Ziel ist selbstständiges und ergebnisoffenes Arbeiten. Ergebnisse sollen dennoch vergleichbar sein. &
%-------------------------------------------------------------------------------------------------------------------------------------------------
SuS kommen zu spät. \newline & 
%-------------------------------------------------------------------------------------------------------------------------------------------------
Visualisierung von Szenario, Fragestellung \& Regeln  für das Rollenspiel.
\\ \midrule%\addlinespace[1ex]
%
% ################################################################################################################################################
%
%\textbf{Erarbeitung I} 
\rotatebox[origin=rC]{90}{\textbf{Erarbeitung I}} & 
%-------------------------------------------------------------------------------------------------------------------------------------------------
 &
%-------------------------------------------------------------------------------------------------------------------------------------------------
Planung des Verfahrens fördern unmittelbar die fokussierte Kompetenz. Die schriftlichen Notizen zum Ablauf des Verfahrens stellen das 
	Arbeitsergebnis/Lernprodukt dar. &
%-------------------------------------------------------------------------------------------------------------------------------------------------
Aufgabenstellung und Material lassen handlungsorientiertes und forschendes Lernen zu. & 
%-------------------------------------------------------------------------------------------------------------------------------------------------
Evtl. wird zu früh nach den Lösungskarten zum Abschreiben verlangt. &
%-------------------------------------------------------------------------------------------------------------------------------------------------
Der Zugriff ist nur am Lehrerpult möglich.
\\ \midrule%\addlinespace[1ex]
%
% ################################################################################################################################################
%
%\textbf{Sicherung I} 
\rotatebox[origin=rC]{90}{\textbf{Sicherung I}} &
%-------------------------------------------------------------------------------------------------------------------------------------------------
Da eine Sicherung der Planung bereits mit Hilfe der Lösungskarten stattgefunden hat, wird hier die Darstellung der Ergebnisse fokussiert. &
%-------------------------------------------------------------------------------------------------------------------------------------------------
Refelexion der  Durch\-füh\-rung ist Bestandteil der Präsentation, Beurteilung der Vermutung ist unverzichtbarer Teil der Erkenntnisgewinnung. & 
%-------------------------------------------------------------------------------------------------------------------------------------------------
& 
%-------------------------------------------------------------------------------------------------------------------------------------------------
Gruppen brauchen lange zum Aufräumen & 
%-------------------------------------------------------------------------------------------------------------------------------------------------
gesteuerte Auswahl der präsentierenden Gruppen
\\ \midrule%\addlinespace[1ex]
%
% ################################################################################################################################################
%
%\textbf{Erarbeitung II}
\rotatebox[origin=rC]{90}{\textbf{Erarbeitung II}} & 
%-------------------------------------------------------------------------------------------------------------------------------------------------
&
%-------------------------------------------------------------------------------------------------------------------------------------------------
&
%-------------------------------------------------------------------------------------------------------------------------------------------------
Aufgabenstellung und Material lassen handlungsorientiertes und forschendes Lernen zu & 
%-------------------------------------------------------------------------------------------------------------------------------------------------
Evtl. wird zu früh nach den Lösungskarten zum abschreiben verlangt. &
%-------------------------------------------------------------------------------------------------------------------------------------------------
Der Zugriff ist nur am Lehrerpult möglich. 
\\ \midrule%\addlinespace[1ex]
%
% ################################################################################################################################################
%
%\textbf{Sicherung II} 
\rotatebox[origin=rC]{90}{\textbf{Sicherung II}}&
%-------------------------------------------------------------------------------------------------------------------------------------------------
&
%-------------------------------------------------------------------------------------------------------------------------------------------------
&
%-------------------------------------------------------------------------------------------------------------------------------------------------
& 
%-------------------------------------------------------------------------------------------------------------------------------------------------
Gruppen brauchen lange zum Aufräumen & 
%-------------------------------------------------------------------------------------------------------------------------------------------------

\\ \addlinespace[1ex]
\bottomrule
\end{longtable}
%
