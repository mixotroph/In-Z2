\begin{longtable}{lp{5cm}X}
		% Caption
		% #######
%		\caption{%
%        	\scriptsize Unterrichtsverlaufplan.%
%        		Abk.: SuS: Schülerinnen und Schüler, AB: Arbeitsblatt.%
%        		} \\[1em] % 
%		%
		% Header (nur) auf der ersten Seite  
		% #################################          
        \toprule%
        \textbf{Std.} &\textbf{Thema/Inhalt der UE} & \textbf{Kompetenz und Standardbezug laut RLP} \\ \midrule% 			
        \endfirsthead%
        %
        % Header auf den Folgeseiten
        % ##########################
        \caption{\footnotesize(Fortzsetzung)}\\ \\%
   		\toprule%
        \textbf{Std.} &\textbf{Thema/Inhalt der UE} & \textbf{Kompetenz und Standardbezug laut RLP} \\ \midrule% 
        \endhead%
		%        
        % Bei Seitenwechsel Tabelle mit einer Linie abschließen
		% #####################################################        
        \bottomrule \addlinespace[0.5em] \multicolumn{3}{c}{(\footnotesize Fortsetzung auf der nächsten Seite)} \\% 
        \endfoot%
		%		
		\bottomrule %
		\endlastfoot%        	


1 & \textbf{Struktur Mobilfunknetz} 
	\begin{itemize}
		\item Aufbau des Mobilfunknetz
		\item Funktion der Komponenten
	\end{itemize} & 
	\textbf{Informatiksysteme verstehen:}Die SuS diskutieren Funktionalität […] von Informatiksystemen.
	 \\ %\addlinespace[.5em]	 
2 & \textbf{Verbindungsaufbau im Mobilfunknetz} 
	 & 
	\textbf{Kommunizieren und Kooperieren:} Die SuS dokumentieren, visualisieren, präsentieren und verteidigen Ergebnisse der Teamarbeit. (S. )
	 \\ \addlinespace[.5em]	 
3-4 & \textbf{\textit{Der ultimativer Abhör\-alpt\-raum}} 
	\begin{itemize}
		\item Sicherheit im Mobilfunknetz
	\end{itemize} & 
	\textbf{Wechselwirkung zwischen Informatiksystemen, Mensch und Gesellschaft:} Die SuS bewerten Risiken und Chancen von Informatiksystemen. (RLP S. 17) \newline
	\textbf{Kommunizieren und Kooperieren} Die SuS verwenden selbstständig Fachtexte […]. (RLP S. 16)
	\\ \addlinespace[.5em]
6 & \textbf{Verschlüsselungsverfahren} 
	\begin{itemize}
		\item 
	\end{itemize} &  
	\textbf{Erkenntnisse gewinnen:} Die SuS können Hypothesen aufstellen, die auf naturwissenschaftlichen Fragestellungen basieren. (2.2.2)\newline
	\textbf{Erkenntnisse gewinnen:} Die SuS können das Untersuchungsergebnis unter Rückbezug auf die Hypothese beschreiben. (2.2.2)
	\\ \addlinespace[.5em]	
\rowcolor{lightgray}
7 & \textbf{Symmetrische Verschlüsselungsverfahren} 
	\begin{itemize}
		\item Asymmetrische Verschlüsselung
		\item Falltürfunktion
	\end{itemize} &  
	\textbf{Problemlösen} Die SuS wenden die Phasen des Problemlöseprozesses (informelle Problembeschreibung, formale Modellierung, Implementierung und Realisierung, Bewertung und Modellkritik) an. RLP (S. 16)
	\\ \addlinespace[.5em]
8 & \textbf{Hybride Verschlüsselungsverfahren} 
	\begin{itemize}
		\item 
	\end{itemize} &  
	\textbf{Problemlösen}\\ %\addlinespace[.5em]		
\end{longtable}