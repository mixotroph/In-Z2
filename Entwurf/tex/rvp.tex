\small
\begin{longtable}{lp{5cm}X}
%\caption{Darstellung der Unterrichtsreihe. Abkürzungen: RLP: Rahmenlehrplan, UE: Unterrichtseinheit}
\toprule
\textbf{Std.} &\textbf{Thema/Inhalt der UE} & \textbf{Kompetenz und Standardbezug} \\
\midrule \endfirsthead
\caption{(Fortzsetzung)}\\ \\
\toprule
\textbf{Std.} &\textbf{Thema/Inhalt der UE} & \textbf{Kompetenz und Standardbezug} \\
\midrule \endhead
1 & \textbf{Tiere im Unterricht:} 
	\begin{itemize}
		\item Merkmale der Hausmaus
		\item (Explorations-)Verhalten allgemein und in einer Arena
	\end{itemize} & 
	\textit{\textbf{Erkenntnisse gewinnen:} Naturwissenschaftliche Untersuchungen durchführen (2.2.2)}\newline
		\textbf{Fragestellung:} Die SuS können naturwissenschaftliche Fragen unter Einbeziehung ihres Fachwissens formulieren.
	 \\ \addlinespace[.5em]
2-4 & \textbf{Planung und Durchführung eines Open-Field-Tests}
	\begin{itemize}
		\item Planung Versuch Wandkontaktverhalten 
	\end{itemize} & 
	\textit{\textbf{Erkenntnisse gewinnen:} Naturwissenschaftliche Untersuchungen durchführen (2.2.2)}\newline
		\textbf{Hypothesenbildung:} Die SuS können aufgestellte Hypothesen bestätigen oder nach Widerlegung weitere Hypothesen entwickeln.
	\\ \addlinespace[.5em]
5 & \begin{itemize}
		\item Besprechung der Ergebnisse
	\end{itemize} &  
	\textit{\textbf{Kommunizieren:} Informationen weitergeben – Textproduktion (mündlich und schriftlich) (2.3.2)}\newline
		\textbf{Auswertung und Reflexion:} Die SuS können Untersuchungsergebnisse (auch erwartungswidrige) interpretieren.
	\\ \addlinespace[.5em] \rowcolor{lightgray}
%6  & \textbf{Brauchen Mäuse Überblick?}
%	\begin{itemize}
%		\item  Durchführung eines zweite, abgeänderten Open-Field-Tests
%	\end{itemize} & 
%	\textit{\textbf{Kommunizieren:} Informationen weitergeben – Textproduktion (mündlich und schriftlich) (2.3.2)}\newline
%		\textbf{Dokumentieren:} Die SuS können Untersuchungen selbstständig protokollieren (F) und anhand des Protokolls den Versuch erläutern (G).
%	\\ \addlinespace[.5em]
6  & \textbf{Brauchen Mäuse Überblick?}
	\begin{itemize}
		\item  Durchführung eines zweite, abgeänderten Open-Field-Tests
	\end{itemize} & 
	\textit{\textbf{Erkenntnisse gewinnen:} Naturwissenschaftliche Untersuchungen durchführen (2.2.2)}\newline
		\textbf{Planung und Durchführung:} Die SuS können Experimente mit Kontrolle planen und durchführen.
	\\ \addlinespace[.5em]
7	& \textbf{Die statistische Auswertung: Der Chi$^2$-Test}
	\begin{itemize}
		\item Alles nur Zufall? Was ist ein Signifikanztest?
		\item Chi$^2$-Test (Referat) mit Übung
	\end{itemize} & 
	\textit{\textbf{Erkenntnisse gewinnen:} Elemente der Mathematik anwenden (2.2.4)}\newline
		Die SuS können vorgegebene Verfahren der Mathematik beim Umgang mit […] Tabellen anwenden.
	\\ \addlinespace[.5em]
8-11 & 	\textbf{Erstellung einer wissenschaftlichen Arbeit}
	\begin{itemize}
		\item Auswertung der Daten mit \textit{Libre Office Calc} (Rohdaten kodieren und Signifikanztest)
		\item Paper schreiben mit \LaTeX
	\end{itemize} &  
	\textit{\textbf{Kommunizieren:} Informationen weitergeben – Textproduktion (mündlich und schriftlich) (2.3.2)}\newline
		\textbf{Texte zu Sachverhalten produzieren:} Die SuS können naturwissenschaftliche Sachverhalte adressaten- und sachgerecht in verschiedenen Darstellungsformen erklären.
	\\ \addlinespace[.5em]
12-13 & 	\textbf{Präsentation der Versuchsergebnisse}
	\begin{itemize}
		\item Aufbau eines wissenschaftlichen Posters
		\item Erstellen des Posters
		\item Postersession
	\end{itemize} & 
	\textit{\textbf{Kommunizieren:} Informationen weitergeben – Textproduktion (mündlich und schriftlich) (2.3.2)}\newline
		\textbf{Präsentieren:} Die SuS können sach-, situations- und adressatenbezogen Untersuchungsmethoden und Ergebnisse präsentieren (F) und 
		Medien für eine Präsentation kriterienorientiert auswählen und die Auswahl reflektieren. \\
\bottomrule
\end{longtable}