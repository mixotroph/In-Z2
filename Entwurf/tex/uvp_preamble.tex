\newenvironment{uvp}{%
	\begin{longtable}{>{\footnotesize}p{2cm}ss>{\footnotesize}p{3cm}}%
		% Caption
		% #######
		\caption{%
        	\scriptsize Unterrichtsverlaufplan.%
        		Abk.: SuS: Schülerinnen und Schüler, AB: Arbeitsblatt.%
        		} \\[1em] % 
		%
		% Header (nur) auf der ersten Seite  
		% #################################          
        \toprule%
        \textbf{Phase/Zeit/\newline Dauer} & \textbf{Geplantes Verhalten der Lehrkraft/\textit{Leitimpulse}} &% 
        	\textbf{Antizipierte Aktivitäten der SuS} &% 
        \textbf{Sozialform/Medien/\newline Materialien}  \\ \midrule% 			
        \endfirsthead%
        %
        % Header auf den Folgeseiten
        % ##########################
        \caption{(Fortzsetzung)}\\ \\%
   		\toprule%
        \textbf{Phase/Zeit/\newline Dauer} &\textbf{Geplantes Verhalten der Lehrkraft/\textit{Leitimpulse}} &% 
        	\textbf{Antizipierte Aktivitäten der SuS}  &% 
        \textbf{Sozialform/Medien/\newline Materialien}  \\ \midrule% 
        \endhead%
		%        
        % Bei Seitenwechsel Tabelle mit einer Linie abschließen
		% #####################################################        
        \bottomrule \addlinespace[0.5em] \multicolumn{4}{c}{(\footnotesize Fortsetzung auf der nächsten Seite)} \\% 
        \endfoot%
		%		
		\bottomrule %
		\endlastfoot%        	

	}{\end{longtable}}%

\newcommand{\Einstieg}[5]{%
	\textbf{Einstieg} \newline #1 \newline\textcolor{red}{#2}  & #3 & #4  & #5\\ \addlinespace[.5em]
}
\newcommand{\Erarbeitung}[5]{%
	 \textbf{Erarbeitung} \newline #1 \newline\textcolor{red}{#2} & #3 & #4  & #5\\ \addlinespace[.5em]
}
\newcommand{\Sicherung}[5]{%
	 \textbf{Sicherung} \newline #1 \newline\textcolor{red}{#2} & #3 & #4  & #5\\ \addlinespace[.5em]
}
\newcommand{\Gelenk}[1]{%
	\midrule
	\multicolumn{4}{p{.95\columnwidth}}{\small \textbf{Gelenkstelle:} #1}\\ \midrule% \addlinespace[.5em]
}